%! Author = Kevin Lin
%! Date = 1/26/2026

% Preamble
\documentclass[11pt,a4paper,margin=1in]{article}

% Packages
\usepackage{amsmath}
\usepackage{amssymb}
\usepackage{enumerate}
\usepackage{pdfpages}

\title{HW 1}
\author{Kevin Lin}
\date{1/26/2026}

% Document
\begin{document}
\maketitle

\section{}
    Let:
    $$
        A = \begin{bmatrix} 
                4 & 1 & 3 & 6 \\ 
                2 & 7 & 5 & 3
            \end{bmatrix},
        B = \begin{bmatrix}
                0 & 4 \\
                7 & 6 \\
                5 & 8 \\
                3 & 11
            \end{bmatrix},
        C = \begin{bmatrix}
                -13 & 0 & 2 \\
                5 & 2 & 10 \\
                0 & 7 & 9
            \end{bmatrix},
        D = \begin{bmatrix}
                5 & -3 & -7 \\
                4 & 0 & 10 \\
                7 & 3 & 11
            \end{bmatrix},
        E = \begin{bmatrix}
                -4 & 6 \\
                12 & 7
            \end{bmatrix}
    $$
    \begin{enumerate}[(a)]
        \item $(3B)^T$:
            \begin{align*}
                (3B)^T &= 3 \cdot B^T \\
                &= 3 \cdot \begin{bmatrix}
                    0 & 7 & 5 & 3 \\
                    4 & 6 & 8 & 11
                \end{bmatrix} \\
                &= \begin{bmatrix}
                    0 & 21 & 15 & 9 \\
                    12 & 18 & 24 & 33
                \end{bmatrix}
            \end{align*}
        \item $(A - B)^T$ is not possible due to dimension mismatch. $A$ is $2 \times 4$
        while $B$ is $4 \times 2$.
        \item $(2B^T - A)^T$:
            \begin{align*}
                (2B^T - A)^T &= 2B - A^T \\
                &= 2 \cdot \begin{bmatrix}
                    0 & 4 \\
                    7 & 6 \\
                    5 & 8 \\
                    3 & 11
                \end{bmatrix} - \begin{bmatrix}
                    4 & 2 \\
                    1 & 7 \\
                    3 & 5 \\
                    6 & 3
                \end{bmatrix} \\
                &= \begin{bmatrix}
                    0 & 8 \\
                    14 & 12 \\
                    10 & 16 \\
                    6 & 22
                \end{bmatrix} - \begin{bmatrix}
                    4 & 2 \\
                    1 & 7 \\
                    3 & 5 \\
                    6 & 3
                \end{bmatrix} \\
                &= \begin{bmatrix}
                    -4 & 6 \\
                    13 & 5 \\
                    7 & 11 \\
                    0 & 19
                \end{bmatrix}
            \end{align*}
        \item $(C + 2D^T + E)^T$ is not possible due to dimension mismatch. $C$
        and $D$ are both $3 \times 3$ while $E$ is $2 \times 2$.
        \item $(-A)^TE$:
            \begin{align*}
                (-A)^TE &= -A^TE \\
                &= -\begin{bmatrix}
                    4 & 2 \\
                    1 & 7 \\
                    3 & 5 \\
                    6 & 3
                \end{bmatrix} \cdot \begin{bmatrix}
                    -4 & 6 \\
                    12 & 7
                \end{bmatrix} \\
                &= -\begin{bmatrix}
                    4 \cdot -4 + 2 \cdot 12 & 4 \cdot 6 + 2 \cdot 7 \\
                    1 \cdot -4 + 7 \cdot 12 & 1 \cdot 6 + 7 \cdot 7 \\
                    3 \cdot -4 + 5 \cdot 12 & 3 \cdot 6 + 5 \cdot 7 \\
                    6 \cdot -4 + 3 \cdot 12 & 6 \cdot 6 + 3 \cdot 7
                \end{bmatrix} \\
                &= -\begin{bmatrix}
                    -16 + 24 & 24 + 14 \\
                    -4 + 84 & 6 + 49 \\
                    -12 + 60 & 18 + 35 \\
                    -24 + 36 & 36 + 21
                \end{bmatrix} \\
                &= -\begin{bmatrix}
                    8 & 38 \\
                    80 & 55 \\
                    48 & 53 \\
                    12 & 57
                \end{bmatrix} \\
                &= \begin{bmatrix}
                    -8 & -38 \\
                    -80 & -55 \\
                    -48 & -53 \\
                    -12 & -57
                \end{bmatrix}
            \end{align*}
    \end{enumerate}

\section{}
    No, $AB \neq BA$. Matrix multiplication is not commutative. We can prove this
    by calculating both $AB$ and $BA$:
    \begin{align*}
        AB &= \begin{bmatrix}
                2 & 7 & 3 \\
                1 & 0 & 9 \\
                -1 & 2 & 10
            \end{bmatrix} \begin{bmatrix}
                -2 & 0 & 3 \\
                2 & -1 & 7 \\
                6 & 4 & -3
            \end{bmatrix} \\
        AB &= \begin{bmatrix}
                2 \cdot -2 + 7 \cdot 2 + 3 \cdot 6 & 2 \cdot 0 + 7 \cdot -1 + 3 \cdot 4 & 2 \cdot 3 + 7 \cdot 7 + 3 \cdot -3 \\
                1 \cdot -2 + 0 \cdot 2 + 9 \cdot 6 & 1 \cdot 0 + 0 \cdot -1 + 9 \cdot 4 & 1 \cdot 3 + 0 \cdot 7 + 9 \cdot -3 \\
                -1 \cdot -2 + 2 \cdot 2 + 10 \cdot 6 & -1 \cdot 0 + 2 \cdot -1 + 10 \cdot 4 & -1 \cdot 3 + 2 \cdot 7 + 10 \cdot -3
            \end{bmatrix} \\
        AB &= \begin{bmatrix}
                -4 + 14 + 18 & 0 - 7 + 12 & 6 + 49 - 9 \\
                -2 + 0 + 54 & 0 + 0 + 36 & 3 + 0 - 27 \\
                2 + 4 + 60 & 0 - 2 + 40 & -3 + 14 - 30
            \end{bmatrix} \\
        AB &= \begin{bmatrix}
                28 & 5 & 46 \\
                52 & 36 & -24 \\
                66 & 38 & -19
            \end{bmatrix} \\
        BA &= \begin{bmatrix}
                -2 & 0 & 3 \\
                2 & -1 & 7 \\
                6 & 4 & -3
            \end{bmatrix} \begin{bmatrix}
                2 & 7 & 3 \\
                1 & 0 & 9 \\
                -1 & 2 & 10
            \end{bmatrix} \\
        BA &= \begin{bmatrix}
                -2 \cdot 2 + 0 \cdot 1 + 3 \cdot -1 & -2 \cdot 7 + 0 \cdot 0 + 3 \cdot 2 & -2 \cdot 3 + 0 \cdot 9 + 3 \cdot 10 \\
                2 \cdot 2 + -1 \cdot 1 + 7 \cdot -1 & 2 \cdot 7 + -1 \cdot 0 + 7 \cdot 2 & 2 \cdot 3 + -1 \cdot 9 + 7 \cdot 10 \\
                6 \cdot 2 + 4 \cdot 1 + -3 \cdot -1 & 6 \cdot 7 + 4 \cdot 0 + -3 \cdot 2 & 6 \cdot 3 + 4 \cdot 9 + -3 \cdot 10
            \end{bmatrix} \\
        BA &= \begin{bmatrix}
                -4 + 0 - 3 & -14 + 0 + 6 & -6 + 0 + 30 \\
                4 - 1 - 7 & 14 + 0 + 14 & 6 - 9 + 70 \\
                12 + 4 + 3 & 42 + 0 - 6 & 18 + 36 - 30
            \end{bmatrix} \\
        BA &= \begin{bmatrix}
                -7 & -8 & 24 \\
                -4 & 28 & 67 \\
                19 & 36 & 24
            \end{bmatrix}
    \end{align*}
    Thus, $AB \neq BA$.
\section{}
    \begin{itemize}
        \item $\ell_1$ norm of $[0, 0, 0]$:
            \[
                ||[0, 0, 0]||_1 = |0| + |0| + |0| = 0
            \]
        $\ell_2$ norm of $[0, 0, 0]$:
            \[
                ||[0, 0, 0]||_2 = \sqrt{0^2 + 0^2 + 0^2} = 0
            \]
        $\ell_\infty$ norm of $[0, 0, 0]$:

        \item $\ell_1$ norm of $[1, 2, 3]$:
            \[
                ||[1, 2, 3]||_1 = |1| + |2| + |3| = 6
            \]
        $\ell_2$ norm of $[1, 2, 3]$:
            \[
                ||[1, 2, 3]||_2 = \sqrt{1^2 + 2^2 + 3^2} = \sqrt{14}
            \]

        \item $\ell_1$ norm of $[2, 4, 6]$:
            \[
                ||[2, 4, 6]||_1 = |2| + |4| + |6| = 12
            \]
        $\ell_2$ norm of $[2, 4, 6]$:
            \[
                ||[2, 4, 6]||_2 = \sqrt{2^2 + 4^2 + 6^2} = \sqrt{56} = 2\sqrt{14}
            \]

        These norms are related to those of $[1, 2, 3]$ by a factor of 2.
        \item 
    \end{itemize}
\end{document}