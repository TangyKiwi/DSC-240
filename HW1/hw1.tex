%! Author = Kevin Lin
%! Date = 1/26/2026

% Preamble
\documentclass[11pt,a4paper,margin=1in]{article}

% Packages
\usepackage{amsmath}
\usepackage{amssymb}
\usepackage{enumerate}
\usepackage{pdfpages}
\usepackage{multirow}
\usepackage{graphicx}

\title{HW 1}
\author{Kevin Lin}
\date{1/26/2026}

% Document
\begin{document}
\maketitle

\section{}
    Let:
    $$
        A = \begin{bmatrix} 
                4 & 1 & 3 & 6 \\ 
                2 & 7 & 5 & 3
            \end{bmatrix},
        B = \begin{bmatrix}
                0 & 4 \\
                7 & 6 \\
                5 & 8 \\
                3 & 11
            \end{bmatrix},
        C = \begin{bmatrix}
                -13 & 0 & 2 \\
                5 & 2 & 10 \\
                0 & 7 & 9
            \end{bmatrix},
        D = \begin{bmatrix}
                5 & -3 & -7 \\
                4 & 0 & 10 \\
                7 & 3 & 11
            \end{bmatrix},
        E = \begin{bmatrix}
                -4 & 5 \\
                12 & 7
            \end{bmatrix}
    $$
    \begin{enumerate}[(a)]
        \item $(3B)^T$:
            \begin{align*}
                (3B)^T &= 3 \cdot B^T \\
                &= 3 \cdot \begin{bmatrix}
                    0 & 7 & 5 & 3 \\
                    4 & 6 & 8 & 11
                \end{bmatrix} \\
                &= \begin{bmatrix}
                    0 & 21 & 15 & 9 \\
                    12 & 18 & 24 & 33
                \end{bmatrix}
            \end{align*}
        \item $(A - B)^T$ is not possible due to dimension mismatch. $A$ is $2 \times 4$
        while $B$ is $4 \times 2$.
        \item $(2B^T - A)^T$:
            \begin{align*}
                (2B^T - A)^T &= 2B - A^T \\
                &= 2 \cdot \begin{bmatrix}
                    0 & 4 \\
                    7 & 6 \\
                    5 & 8 \\
                    3 & 11
                \end{bmatrix} - \begin{bmatrix}
                    4 & 2 \\
                    1 & 7 \\
                    3 & 5 \\
                    6 & 3
                \end{bmatrix} \\
                &= \begin{bmatrix}
                    0 & 8 \\
                    14 & 12 \\
                    10 & 16 \\
                    6 & 22
                \end{bmatrix} - \begin{bmatrix}
                    4 & 2 \\
                    1 & 7 \\
                    3 & 5 \\
                    6 & 3
                \end{bmatrix} \\
                &= \begin{bmatrix}
                    -4 & 6 \\
                    13 & 5 \\
                    7 & 11 \\
                    0 & 19
                \end{bmatrix}
            \end{align*}
        \item $(C + 2D^T + E)^T$ is not possible due to dimension mismatch. $C$
        and $D$ are both $3 \times 3$ while $E$ is $2 \times 2$.
        \item $(-A)^TE$:
            \begin{align*}
                (-A)^TE &= -A^TE \\
                &= -\begin{bmatrix}
                    4 & 2 \\
                    1 & 7 \\
                    3 & 5 \\
                    6 & 3
                \end{bmatrix} \cdot \begin{bmatrix}
                    -4 & 5 \\
                    12 & 7
                \end{bmatrix} \\
                &= -\begin{bmatrix}
                    4 \cdot -4 + 2 \cdot 12 & 4 \cdot 5 + 2 \cdot 7 \\
                    1 \cdot -4 + 7 \cdot 12 & 1 \cdot 5 + 7 \cdot 7 \\
                    3 \cdot -4 + 5 \cdot 12 & 3 \cdot 5 + 5 \cdot 7 \\
                    6 \cdot -4 + 3 \cdot 12 & 6 \cdot 5 + 3 \cdot 7
                \end{bmatrix} \\
                &= -\begin{bmatrix}
                    -16 + 24 & 20 + 14 \\
                    -4 + 84 & 5 + 49 \\
                    -12 + 60 & 15 + 35 \\
                    -24 + 36 & 30 + 21
                \end{bmatrix} \\
                &= -\begin{bmatrix}
                    8 & 34 \\
                    80 & 54 \\
                    48 & 50 \\
                    12 & 51
                \end{bmatrix} \\
                &= \begin{bmatrix}
                    -8 & -34 \\
                    -80 & -54 \\
                    -48 & -50 \\
                    -12 & -51
                \end{bmatrix}
            \end{align*}
    \end{enumerate}

\section{}
    No, $AB \neq BA$. Matrix multiplication is not commutative. We can prove this
    by calculating both $AB$ and $BA$:
    \begin{align*}
        AB &= \begin{bmatrix}
                2 & 7 & 3 \\
                1 & 0 & 9 \\
                -1 & 2 & 10
            \end{bmatrix} \begin{bmatrix}
                -2 & 0 & 3 \\
                2 & -1 & 7 \\
                6 & 4 & -3
            \end{bmatrix} \\
        AB &= \begin{bmatrix}
                2 \cdot -2 + 7 \cdot 2 + 3 \cdot 6 & 2 \cdot 0 + 7 \cdot -1 + 3 \cdot 4 & 2 \cdot 3 + 7 \cdot 7 + 3 \cdot -3 \\
                1 \cdot -2 + 0 \cdot 2 + 9 \cdot 6 & 1 \cdot 0 + 0 \cdot -1 + 9 \cdot 4 & 1 \cdot 3 + 0 \cdot 7 + 9 \cdot -3 \\
                -1 \cdot -2 + 2 \cdot 2 + 10 \cdot 6 & -1 \cdot 0 + 2 \cdot -1 + 10 \cdot 4 & -1 \cdot 3 + 2 \cdot 7 + 10 \cdot -3
            \end{bmatrix} \\
        AB &= \begin{bmatrix}
                -4 + 14 + 18 & 0 - 7 + 12 & 6 + 49 - 9 \\
                -2 + 0 + 54 & 0 + 0 + 36 & 3 + 0 - 27 \\
                2 + 4 + 60 & 0 - 2 + 40 & -3 + 14 - 30
            \end{bmatrix} \\
        AB &= \begin{bmatrix}
                28 & 5 & 46 \\
                52 & 36 & -24 \\
                66 & 38 & -19
            \end{bmatrix} \\
        BA &= \begin{bmatrix}
                -2 & 0 & 3 \\
                2 & -1 & 7 \\
                6 & 4 & -3
            \end{bmatrix} \begin{bmatrix}
                2 & 7 & 3 \\
                1 & 0 & 9 \\
                -1 & 2 & 10
            \end{bmatrix} \\
        BA &= \begin{bmatrix}
                -2 \cdot 2 + 0 \cdot 1 + 3 \cdot -1 & -2 \cdot 7 + 0 \cdot 0 + 3 \cdot 2 & -2 \cdot 3 + 0 \cdot 9 + 3 \cdot 10 \\
                2 \cdot 2 + -1 \cdot 1 + 7 \cdot -1 & 2 \cdot 7 + -1 \cdot 0 + 7 \cdot 2 & 2 \cdot 3 + -1 \cdot 9 + 7 \cdot 10 \\
                6 \cdot 2 + 4 \cdot 1 + -3 \cdot -1 & 6 \cdot 7 + 4 \cdot 0 + -3 \cdot 2 & 6 \cdot 3 + 4 \cdot 9 + -3 \cdot 10
            \end{bmatrix} \\
        BA &= \begin{bmatrix}
                -4 + 0 - 3 & -14 + 0 + 6 & -6 + 0 + 30 \\
                4 - 1 - 7 & 14 + 0 + 14 & 6 - 9 + 70 \\
                12 + 4 + 3 & 42 + 0 - 6 & 18 + 36 - 30
            \end{bmatrix} \\
        BA &= \begin{bmatrix}
                -7 & -8 & 24 \\
                -4 & 28 & 67 \\
                19 & 36 & 24
            \end{bmatrix}
    \end{align*}
    Thus, $AB \neq BA$.

\section{}
    \begin{itemize}
        \item $\ell_1$ norm of $[0, 0, 0]$:
            \[
                ||[0, 0, 0]||_1 = |0| + |0| + |0| = 0
            \]
        $\ell_2$ norm of $[0, 0, 0]$:
            \[
                ||[0, 0, 0]||_2 = \sqrt{0^2 + 0^2 + 0^2} = 0
            \]
        $\ell_\infty$ norm of $[0, 0, 0]$:
            \[
                ||[0, 0, 0]||_\infty = \max(|0|, |0|, |0|) = 0
            \]
        \item $\ell_1$ norm of $[1, 2, 3]$:
            \[
                ||[1, 2, 3]||_1 = |1| + |2| + |3| = 6
            \]
        $\ell_2$ norm of $[1, 2, 3]$:
            \[
                ||[1, 2, 3]||_2 = \sqrt{1^2 + 2^2 + 3^2} = \sqrt{14}
            \]
        $\ell_\infty$ norm of $[1, 2, 3]$:
            \[
                ||[1, 2, 3]||_\infty = \max(|1|, |2|, |3|) = 3
            \]
        \item $\ell_1$ norm of $[2, 4, 6]$:
            \[
                ||[2, 4, 6]||_1 = |2| + |4| + |6| = 12
            \]
        $\ell_2$ norm of $[2, 4, 6]$:
            \[
                ||[2, 4, 6]||_2 = \sqrt{2^2 + 4^2 + 6^2} = \sqrt{56} = 2\sqrt{14}
            \]
        $\ell_\infty$ norm of $[2, 4, 6]$:
            \[
                ||[2, 4, 6]||_\infty = \max(|2|, |4|, |6|) = 6
            \]
        These norms are related to those of $[1, 2, 3]$ by a factor of 2.
        \item No, a vector can not have a negative norm by definition as the norm
        is positive definite.
    \end{itemize}

\section{} 
    Given $X = [x_1, x_2, \dots, x_n] \in \mathbb{R}^{m\times n}$ where $x_i \in \mathbb{R}^m$
    for all $i$, and $Y^T = [y_1, y_2, \dots, y_n] \in \mathbb{R}^{p\times n}$ where
    $y_i \in \mathbb{R}^p$ for all $i$, $XY = \sum_{i=1}^{n} x_i y_i^T$.
    We can prove this as follows:
    \begin{align*}
        XY &= X \cdot Y \\
        &= [x_1, x_2, \dots, x_n] \cdot \begin{bmatrix}
            y_1^T \\
            y_2^T \\
            \vdots \\
            y_n^T
        \end{bmatrix} \\
        &= x_1 y_1^T + x_2 y_2^T + \dots + x_n y_n^T \\
        &= \sum_{i=1}^{n} x_i y_i^T
    \end{align*}

\section{}
    Given $X \in \mathbb{R}^{m\times n}$, we can prove $X^TX$ is symmetric and 
    positive semi-definite as follows:
    \begin{itemize}
        \item Symmetric:
            \begin{align*}
                (X^TX)^T &= X^T(X^T)^T \\
                &= X^TX
            \end{align*}
        \item Positive semi-definite:
            For any non-zero vector $z \in \mathbb{R}^n$,
            \begin{align*}
                z^T(X^TX)z &= (Xz)^T(Xz) \\
                &= ||Xz||_2^2 \\
                &\geq 0
            \end{align*}
    \end{itemize}

\section{}
    Given $g(x, y) = e^{(x + y)} + e^{3xy} + e^{y^4}$, we can compute the partial derivatives
    as follows:
    \begin{align*}
        \frac{\partial g}{\partial x} &= \frac{\partial}{\partial x} \left( e^{(x + y)} + e^{3xy} + e^{y^4} \right) \\
        &= e^{(x + y)} + 3y e^{3xy} + 0 \\
        &= e^{(x + y)} + 3y e^{3xy} \\
        \frac{\partial g}{\partial y} &= \frac{\partial}{\partial y} \left( e^{(x + y)} + e^{3xy} + e^{y^4} \right) \\
        &= e^{(x + y)} + 3x e^{3xy} + 4y^3 e^{y^4} \\
        &= e^{(x + y)} + 3x e^{3xy} + 4y^3 e^{y^4}
    \end{align*}

\section{}
    Let $A = \begin{bmatrix} 1 & 4 \\ 2 & 3 \end{bmatrix}$.
    \begin{enumerate}[(a)]
        \item We compute the eigenvalues and eigenvectors as follows:
        \begin{align*}
            \text{det}(A - \lambda I) &= 0 \\
            \text{det}\begin{bmatrix}
                1 - \lambda & 4 \\
                2 & 3 - \lambda
            \end{bmatrix} &= 0 \\
            (1 - \lambda)(3 - \lambda) - 8 &= 0 \\
            \lambda^2 - 4\lambda - 5 &= 0 \\
            (\lambda - 5)(\lambda + 1) &= 0 \\
            \lambda_1 &= 5, \quad \lambda_2 = -1
        \end{align*}
        For $\lambda_1 = 5$:
        \begin{align*}
            (A - 5I)v &= 0 \\
            \begin{bmatrix}
                -4 & 4 \\
                2 & -2
            \end{bmatrix} \begin{bmatrix}
                v_1 \\
                v_2
            \end{bmatrix} &= 0 \\
            -4v_1 + 4v_2 &= 0 \\
            v_1 &= v_2
        \end{align*}
        Thus, one eigenvector corresponding to $\lambda_1 = 5$ is $v_1 = \begin{bmatrix} 1 \\ 1 \end{bmatrix}$. \\
        For $\lambda_2 = -1$:
        \begin{align*}
            (A + I)v &= 0 \\
            \begin{bmatrix}
                2 & 4 \\
                2 & 4
            \end{bmatrix} \begin{bmatrix}
                v_1 \\
                v_2
            \end{bmatrix} &= 0 \\
            2v_1 + 4v_2 &= 0 \\
            v_1 &= -2v_2
        \end{align*}
        Thus, one eigenvector corresponding to $\lambda_2 = -1$ is $v_2 = \begin{bmatrix} -2 \\ 1 \end{bmatrix}$.
        \item The eigen-decomposition of $A$ is then:
        \begin{align*}
            A &= PDP^{-1} \\
            P &= \begin{bmatrix}
                1 & -2 \\
                1 & 1
            \end{bmatrix}, \quad D = \begin{bmatrix}
                5 & 0 \\
                0 & -1
            \end{bmatrix} \\
            A &= \begin{bmatrix}
                1 & -2 \\
                1 & 1
            \end{bmatrix} \begin{bmatrix}
                5 & 0 \\
                0 & -1
            \end{bmatrix} \begin{bmatrix}
                1 & -2 \\
                1 & 1
            \end{bmatrix}^{-1}
        \end{align*}
        \item $A$ has rank 2 since its two eigenvalues are non-zero.
        \item $A$ is not positive definite since it has a negative eigenvalue ($\lambda_2 = -1$).
        \item $A$ is not positive semi-definite since it has a negative eigenvalue ($\lambda_2 = -1$).
        \item $A$ is not singular as none of its eigenvalues are zero. Its determinant is:
        \begin{align*}
            \text{det}(A) &= \lambda_1 \cdot \lambda_2 \\
            &= 5 \cdot -1 \\
            &= -5 \neq 0
        \end{align*} which also shows it is not singular.
    \end{enumerate}

\section{}
    We have linear classifier $h(x) = \text{sign}(w^Tx)$ where $w = [w_0, w_1, w_2]^T$
    and $x = [1, x_1, x_2]^T$.
    \begin{enumerate}[(a)]
        \item We can prove the regions where $h(x) = +1$ and $h(x) = -1$ are separated
        by a line as follows:
        \begin{align*}
            h(x) &= \text{sign}(w^Tx) \\
            &= \text{sign}(w_0 + w_1x_1 + w_2x_2)
        \end{align*}
        This equation is linear in nature, and thus forms a linear decision boundary
        separating the two regions. \\
        Expressing the line as $x_2 = ax_1 + b$, the slope $a$ and intercept $b$
        in terms of $w_0, w_1, w_2$ are:
        \begin{align*}
            w_0 + w_1x_1 + w_2x_2 &= 0 \\
            w_2x_2 &= -w_0 - w_1x_1 \\
            x_2 &= -\frac{w_1}{w_2}x_1 - \frac{w_0}{w_2} \therefore \\
            a &= -\frac{w_1}{w_2}, \quad b = -\frac{w_0}{w_2}
        \end{align*}
    \end{enumerate}

\section{}
    See attached Jupyter notebook.

\section{}
    \begin{enumerate}[(a)]
        \item Because we are given the blood pressure for 10k patients in the 
        dataset and corresponding data related to each patient, this is a 
        supervised learning regression problem.
        \item The label space consists of two values, the systolic and diastolic
        blood pressure of each patient $(y_\text{systolic}, y_\text{dystolic}) \in \mathbb{R}^2$.
        \item The output space is the same as the label space, as the ML system
        aims to predict both the systolic and dystolic blood pressure values.
    \end{enumerate}

\section{}
    \begin{enumerate}[(a)]
        \item 
            \begin{tabular}{l|l|c|c|}
                \multicolumn{2}{c}{} & \multicolumn{2}{c}{Truth} \\
                \cline{3-4}
                \multicolumn{2}{c|}{} & Positive (Spam) & Negative (Not Spam) \\
                \cline{2-4}
                \multirow{2}{*}{Test}
                    & Positive (Spam) & 1750 & 250 \\
                \cline{2-4}
                    & Negative (Not Spam) & 250 & 7750 \\
                \cline{2-4}
            \end{tabular}
        \item The false positive rate of the system is $250/8000 = 0.03125$.
        \item The false negative rate of the system is $250/2000 = 0.125$.
        \item The error rate of the system is $(250 + 250) / 10000 = 0.05$.
        \item The precision of the system is $1750 / (1750 + 250) = 0.875$.
        \item The sensitivity of the system is $1750 / (1750 + 250) = 0.875$.
    \end{enumerate}

\includepdf[pages=-]{hw1_code.pdf}

\end{document}